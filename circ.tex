\documentclass{article}

\usepackage{libertine}
\renewcommand*\familydefault{\sfdefault}

\usepackage[
             %paper=a4paper,
	     paperheight=1070mm,
	     paperwidth=210mm,
             margin=1cm,
             noheadfoot
            ]{geometry}
 
\usepackage{tikz}
\usepackage[siunitx]{circuitikz}

\begin{document}
\pagestyle{empty}
\section*{Monopoles}

\begin{figure}[!h]
\centering
\begin{circuitikz}
\draw   (0,0) node[ground] (m) {};
\draw   (2,0) node[rground] (m) {};
\draw   (4,0) node[sground] (m) {};
\draw   (6,0) node[tground] (m) {};
\draw   (8,0) node[nground] (m) {};
\draw   (10,0) node[pground] (m) {};
\draw   (12,0) node[cground] (m) {};
\draw   (14,0) node[tlinestub] (m) {};

\draw   (0,-4) node[vcc] (m) {};
\draw   (2,-4) node[vee] (m) {};
\draw   (4,-4) node[match] (m) {};

\draw   (6,-4) node[antenna] (m) {};
\draw   (8,-4) node[rxantenna] (m) {};
\draw   (12,-4) node[txantenna] (m) {};
\end{circuitikz}
\end{figure}

\section*{Bipoles}

\subsection*{instruments}
\begin{figure}[!h]
\centering
\begin{circuitikz}
\draw  (0,0)   to[ammeter] ++(2,0);
\draw  (2.5,0) to[voltmeter] ++(2,0);
\draw  (5,0)   to[ohmmeter] ++(2,0);
\end{circuitikz}
\end{figure}

\subsection*{Basic resistive bipoles}
\begin{figure}[!h]
\centering
\begin{circuitikz}
\draw  (0,0)    to[lamp] ++(2,0);
\draw  (2.5,0)  to[generic] ++(2,0);
\draw  (5,0)    to[tgeneric] ++(2,0);
\draw  (7.5,0)  to[ageneric] ++(2,0);
\draw  (10,0)   to[fullgeneric] ++(2,0);
\draw  (12.5,0) to[tfullgeneric] ++(2,0);
\draw  (15,0)   to[memristor] ++(2,0);
\end{circuitikz}
\end{figure}

\subsection*{Resistors and the like}
\begin{figure}[!h]
\centering
\begin{circuitikz}
\draw  (0,0)    to[R] ++(2,0);
\draw  (2.5,0)  to[vR] ++(2,0);
\draw  (5,0)    to[pR] ++(2,0);
\draw  (7.5,0)  to[european resistor] ++(2,0);
\draw  (10,0)   to[variable european resistor] ++(2,0);
\draw  (12.5,0) to[european potentiometer] ++(2,0);
\draw  (15,0)   to[varistor] ++(2,0);

\draw  (0,-2)    to[photoresistor] ++(2,0);
\draw  (2.5,-2)  to[thermocouple] ++(2,0);
\draw  (5,-2)    to[thermistor] ++(2,0);
\draw  (7.5,-2)  to[thermistor ptc] ++(2,0);
\draw  (10,-2)   to[thermistor ntc] ++(2,0);
\draw  (12.5,-2) to[fuse] ++(2,0);
\draw  (15,-2)   to[afuse] ++(2,0);
\end{circuitikz}
\end{figure}

\subsection*{Diodes and such}
\begin{figure}[!h]
\centering
\begin{circuitikz}
\draw  (0,0)    to[empty diode] ++(2,0);
\draw  (2.5,0)  to[empty Schottky diode] ++(2,0);
\draw  (5,0)    to[empty Zener diode] ++(2,0);
\draw  (7.5,0)  to[empty ZZener diode] ++(2,0);
\draw  (10,0)   to[empty tunnel diode] ++(2,0);
\draw  (12.5,0) to[empty photodiode] ++(2,0);
\draw  (15,0)   to[empty led] ++(2,0);

\draw  (0,-2)    to[empty varcap] ++(2,0);
\draw  (2.5,-2)  to[full diode] ++(2,0);
\draw  (5,-2)    to[full Schottky diode] ++(2,0);
\draw  (7.5,-2)  to[full Zener diode] ++(2,0);
\draw  (10,-2)   to[full ZZener diode] ++(2,0);
\draw  (12.5,-2) to[full tunnel diode] ++(2,0);
\draw  (15,-2)   to[full photodiode] ++(2,0);

\draw  (0,-4)    to[full led] ++(2,0);
\draw  (2.5,-4)  to[full varcap] ++(2,0);
\draw  (5,-4)    to[stroke diode] ++(2,0);
\draw  (7.5,-4)  to[stroke Schottky diode] ++(2,0);
\draw  (10,-4)   to[stroke Zener diode] ++(2,0);
\draw  (12.5,-4) to[stroke ZZener diode] ++(2,0);
\draw  (15,-4)   to[stroke tunnel diode] ++(2,0);

\draw  (0,-6)    to[stroke photodiode] ++(2,0);
\draw  (2.5,-6)  to[stroke led] ++(2,0);
\draw  (5,-6)    to[stroke varcap] ++(2,0);
\end{circuitikz}
\end{figure}

\subsection*{Other tripole-like diodes}
\begin{figure}[!h]
\centering
\begin{circuitikz}
\draw  (0,0)    to[empty triac] ++(2,0);
\draw  (2.5,0)  to[full triac] ++(2,0);
\draw  (5,0)    to[empty thyristor] ++(2,0);
\draw  (7.5,0)  to[full thyristor] ++(2,0);
\draw  (10,0)   to[stroke thyristor] ++(2,0);
\draw  (12.5,0) to[squid] ++(2,0);
\draw  (15,0)   to[barrier] ++(2,0);

\draw  (0,-2)    to[european gas filled surge arrester] ++(2,0);
\draw  (2.5,-2)  to[american gas filled surge arrester] ++(2,0);
\end{circuitikz}
\end{figure}

\subsection*{Basic dynamical bipoles}
\begin{figure}[!h]
\centering
\begin{circuitikz}
\draw  (0,0)    to[capacitor] ++(2,0);
\draw  (2.5,0)  to[polar capacitor] ++(2,0);
\draw  (5,0)    to[ecapacitor] ++(2,0);
\draw  (7.5,0)  to[variable capacitor] ++(2,0);
\draw  (10,0)   to[piezoelectric] ++(2,0);
\draw  (12.5,0) to[cute inductor] ++(2,0);
\draw  (15,0)   to[variable cute inductor] ++(2,0);

\draw  (0,-2)    to[american inductor] ++(2,0);
\draw  (2.5,-2)  to[variable american inductor] ++(2,0);
\draw  (5,-2)    to[european inductor] ++(2,0);
\draw  (7.5,-2)  to[variable european inductor] ++(2,0);
\draw  (10,-2)   to[transmission line] ++(2,0);
\end{circuitikz}
\end{figure}

\subsection*{Stationary sources}
\begin{figure}[!h]
\centering
\begin{circuitikz}
\draw  (0,0)    to[battery] ++(2,0);
\draw  (2.5,0)  to[battery1] ++(2,0);
\draw  (5,0)    to[battery2] ++(2,0);
\draw  (7.5,0)  to[european voltage source] ++(2,0);
\draw  (10,0)   to[american voltage source] ++(2,0);
\draw  (12.5,0) to[european current source] ++(2,0);
\draw  (15,0)   to[american current source] ++(2,0);
\end{circuitikz}
\end{figure}

\subsection*{Sinusoidal sources}
\begin{figure}[!h]
\centering
\begin{circuitikz}
\draw  (0,0)    to[sinusoidal voltage source] ++(2,0);
\draw  (2.5,0)  to[sinusoidal current source,] ++(2,0);
\end{circuitikz}
\end{figure}

\subsection*{Special sources}
\begin{figure}[!h]
\centering
\begin{circuitikz}
\draw  (0,0)    to[square voltage source] ++(2,0);
\draw  (2.5,0)  to[vsourcetri] ++(2,0);
\draw  (5,0)    to[esource] ++(2,0);
\draw  (7.5,0)  to[pvsource] ++(2,0);
\draw  (10,0)   to[ioosource] ++(2,0);
\draw  (12.5,0) to[voosource] ++(2,0);
\end{circuitikz}
\end{figure}

\subsection*{DC sources}
\begin{figure}[!h]
\centering
\begin{circuitikz}
\draw  (0,0)    to[dcvsource] ++(2,0);
\draw  (2.5,0)  to[dcisource] ++(2,0);
\end{circuitikz}
\end{figure}

\subsection*{Mechanical Analogy}
\begin{figure}[!h]
\centering
\begin{circuitikz}
\draw  (0,0)    to[damper] ++(2,0);
\draw  (2.5,0)  to[spring] ++(2,0);
\draw  (5,0)    to[mass] ++(2,0);
\end{circuitikz}
\end{figure}

\subsection*{Switch}
\begin{figure}[!h]
\centering
\begin{circuitikz}
\draw  (0,0)    to[closing switch] ++(2,0);
\draw  (2.5,0)  to[opening switch] ++(2,0);
\draw  (5,0)    to[normal open switch] ++(2,0);
\draw  (7.5,0)  to[normal closed switch] ++(2,0);
\draw  (10,0)   to[push button] ++(2,0);
\end{circuitikz}
\end{figure}

\subsection*{Block diagram components}
\begin{figure}[!h]
\centering
\begin{circuitikz}
\draw  (0,0)    to[twoport] ++(2,0);
\draw  (2.5,0)  to[vco] ++(2,0);
\draw  (5,0)    to[bandpass] ++(2,0);
\draw  (7.5,0)  to[bandstop] ++(2,0);
\draw  (10,0)   to[highpass] ++(2,0);
\draw  (12.5,0) to[lowpass] ++(2,0);
\draw  (15,0)   to[adc] ++(2,0);

\draw  (0,-2)    to[dac] ++(2,0);
\draw  (2.5,-2)  to[dsp] ++(2,0);
\draw  (5,-2)    to[fft] ++(2,0);
\draw  (7.5,-2)  to[amp] ++(2,0);
\draw  (10,-2)   to[vamp] ++(2,0);
\draw  (12.5,-2) to[piattenuator] ++(2,0);
\draw  (15,-2)   to[vpiattenuator] ++(2,0);

\draw  (0,-4)    to[tattenuator] ++(2,0);
\draw  (2.5,-4)  to[vtattenuator] ++(2,0);
\draw  (5,-4)    to[phaseshifter] ++(2,0);
\draw  (7.5,-4)  to[vphaseshifter] ++(2,0);
\draw  (10,-4)   to[detector] ++(2,0);
\end{circuitikz}
\end{figure}

\section*{Tripoles}
\subsection*{Controlled sources}
\begin{figure}[!h]
\centering
\begin{circuitikz}
\draw  (0,0)    to[european controlled voltage source] ++(2,0);
\draw  (2.5,0)  to[american controlled voltage source] ++(2,0);
\draw  (5,0)    to[european controlled current source] ++(2,0);
\draw  (7.5,0)  to[american controlled current source] ++(2,0);
\draw  (10,0)   to[controlled sinusoidal voltage source] ++(2,0);
\draw  (12.5,0) to[controlled sinusoidal current source] ++(2,0);
\end{circuitikz}
\end{figure}

\subsection*{Transistors}
\begin{figure}[!h]
\centering
\begin{circuitikz}
\draw  (0,0)    node[nmos]{};
\draw  (2.5,0)  node[pmos]{};
\draw  (5,0)    node[pmos,emptycircle]{};
\draw  (7.5,0)  node[npn]{};
\draw  (10,0)   node[pnp]{};
\draw  (12.5,0) node[npn,photo]{};
\draw  (15,0)   node[pnp,photo]{};

\draw  (0,-2)    node[nigbt]{};
\draw  (2.5,-2)  node[pigbt]{};
\draw  (5,-2)    node[Lnigbt]{};
\draw  (7.5,-2)  node[Lpigbt]{};
\draw  (10,-2)   node[npn,bodydiode]{};
\draw  (12.5,-2) node[pnp,bodydiode]{};
\draw  (15,-2)   node[nigbt,bodydiode]{};

\draw  (0,-4)    node[pigbt,bodydiode]{};
\draw  (2.5,-4)  node[nfet,bodydiode]{};
\draw  (5,-4)    node[pfet,bodydiode]{};
\draw  (7.5,-4)  node[nfet]{};
\draw  (10,-4)   node[pfet]{};
\draw  (12.5,-4) node[nigfete,solderdot]{};
\draw  (15,-4)   node[nigfetebulk]{};

\draw  (0,-6)    node[nigfetd]{};
\draw  (2.5,-6)  node[pfet]{};
\draw  (5,-6)    node[pigfete]{};
\draw  (7.5,-6)  node[pigfetebulk]{};
\draw  (10,-6)   node[pigfetd]{};
\draw  (12.5,-6) node[njfet]{};
\draw  (15,-6)   node[pjfet]{};

\draw  (0,-8)    node[isfet]{};
\end{circuitikz}
\end{figure}

\subsection*{Electronic Tubes}
\begin{figure}[!h]
\centering
\begin{circuitikz}
\draw  (0,0)    node[magnetron]{};
\end{circuitikz}
\end{figure}

\subsection*{Block diagram}
\begin{figure}[!h]
\centering
\begin{circuitikz}
\draw  (0,0)    node[mixer]{};
\draw  (2.5,0)  node[adder]{};
\draw  (5,0)    node[oscillator]{};
\draw  (7.5,0)  node[circulator]{};
\draw  (10,0)   node[wilkinson]{};
\end{circuitikz}
\end{figure}

\subsection*{Switch}
\begin{figure}[!h]
\centering
\begin{circuitikz}
\draw  (0,0)    node[spdt]{};
\draw  (2.5,0)  node[toggle switch]{};
\end{circuitikz}
\end{figure}

\subsection*{Electro-Mechanical Devices}
\begin{figure}[!h]
\centering
\begin{circuitikz}
\draw  (0,0)    node[elmech]{M};
\draw  (2.5,0)  node[elmech]{G};
\draw  (5,0)    node[elmech]{};
\end{circuitikz}
\end{figure}

\section*{Double bipoles}
\subsection*{Transformers}
\begin{figure}[!h]
\centering
\begin{circuitikz}
\draw  (0,0)    node[transformer]{};
\draw  (3,0)  node[transformer core]{};
\end{circuitikz}
\hspace{1cm}
\begin{circuitikz}[american]
\draw  (0,0)    node[transformer]{};
\draw  (3,0)  node[transformer core]{};
\end{circuitikz}
\hspace{1cm}
\begin{circuitikz}[european]
\draw  (0,0)    node[transformer]{};
\draw  (3,0)  node[transformer core]{};
\end{circuitikz}
\end{figure}
\begin{figure}[!h]
\centering
\begin{circuitikz}
\draw  (0,0)    node[gyrator]{};
\draw  (3,-1)  node[coupler]{};
\draw  (6,-1)    node[coupler2]{};
\end{circuitikz}
\end{figure}

\subsection*{Amplifiers}
\begin{figure}[!h]
\centering
\begin{circuitikz}
\draw  (0,0)  node[op amp]{};
\draw  (3,0)  node[en amp]{};
\draw  (6,0)  node[fd op amp]{};
\draw  (9,0)  node[gm amp]{};
\draw  (12,0) node[plain amp]{};
\draw  (15,0) node[buffer]{};
\end{circuitikz}
\end{figure}

\end{document}

